\section{Plenary Sessions}

\subsection{1st: The nature, causes and consequences of innovation}

Innovation is lead by uncertainty. Nobody knows what are the outcomes of an innovation.
Bringing up a new product or service is also an exercise of guessing. 
Due to the novelty nature of innovation, designing a business plan for a new idea implies into propose solutions for an unknown future outcome. 
The probability of fails tends to be high given the lack of previous knowledge about the market (including costumer and user) reaction regarding the new idea.

There is not a widely accepted definition for innovation, but some characteristics are shared among them. x,y,z . Oslo manual.

The attempt to conceptualize innovation was the ground preparation to introduce the main topic of the talk: the seven lesson about innovation. 
Started by presenting the models of innovation, in particular the linear model, the chain linked model and the "organization model". 
Those were presented in a chronological order, to demonstrate evolution of the comprehension about the factors that influences in innovation, especially how to innovate. 

The second lesson was about sources of innovation. 
It leaded the session to the third lesson, which states that the speed and the nature of innovation depends on the source. 
For example, "aircraft industry against other industry". 
Novelties are new (invention, innovation) combinations of old solutions. 

Innovation does not come like manna from the heaven and is likely to fail. 
The speed and nature of innovation come from the ability to make profit from it. 
Schumpeterian competition.

The conclusion lead to the fact that innovation is hard, expensive, time consuming and prone to fail. But, necessary since our society do not know another way to progress.



