\section{Plenary Sessions}

% 1st plenary: The nature, causes and consequences of innovation

Practical and conceptual aspects of innovation were the focus of the first plenary session. 
Dr. Magnus Holm\'{e}n, explored the topic by giving seven lessons about innovation, which goes from the known innovation models, passing by the sources of innovation and reaching the fact that the speed and nature of innovation come from the ability to make profit from it. 
An general overview of the lessons indicates that innovation is hard, expensive, time consuming and prone to fail; but necessary, since our society do not know another way to progress.

% 2nd plenary session

The second plenary sessions was dedicated to the research project UCARE, which aims at developing a internet-based platform for CBT (Cognitive Behavioral Therapy), in short iCBT. 
In the first half, Dr. Erik Olsson presented the general aspects of the project UCARE, challenges and lessons learned during the ongoing project. 
One of the major drawbacks of internet-based CBTs is the low adherence rate. 
Given the similarities between iCBTs and e-learning platforms, the content delivered to the user may affect the performance of effectiveness of the system. 
The use of static texts (including PDF files), questionnaires and discussion forums in the first versions of UCARE showed that the lack of interactive content would be one the reasons for the low adherence rate of the system.

For the second half, Dr. Helena Grönqvist pointed out the challenges of involving the users into the research process. 
An example given was the experience in the project \textit{TeenCan}, 10 adolescents, including 4 with live experienced of cancer, between 13-19 years old were involved actively involved into the project . 
The recruit process was particularly challenging. 
Only tend subjects attended the discussion meetings, where another challenge raised, the fact that the younger subjects (around 13 and 14 years old) did not fell comfortable to interact to the older subjects (around 18 and 19 years old). 
Additionally, Helena highlighted that the type user involvement strongly depends on the research project.

% 3rd plenary session

The third plenary session had a different dynamics compared to the previous ones. 
Two distinct subjects were addressed: (i) self-motivation as an strong aspect of behavior change and maintenance; (ii) use of innovation theory in applied. 
For the first subject, Dr. Geoffrey Williams, from Rochester University, related behavior change and maintenance with self-motivation theory. 
Some studies were he was involved (Smoker’s Health Study, Physician Work Motivation and Alcohol Use at University) indicate the importance of personal internalization rather than physician reinforcement for behavior change and maintenance.
For a lasting and effective change, the decision and effort should come from the own patient. 
In that sense, the physician should be seem as a consultant rather than an intervenor, by promoting patient autonomy (strongly related to freedom of choice).

Aftermost, Dr. Lars Elbaek and Erik Zijdemans gave an overview of how innovation theory has been used to burst the results of the applied research projects in the Southern Denmark University (SDU), in particular the case of the Department of Sports Science and Clinical Biomechanics (DSSCB). 
As highlighted by them, the most valuable outcome of the University is knowledge, not only from the education, but also research perspective.
In that sense, popular tools for business modeling such as Business Model Canvas, Lean Canvas and Theory U can be also used into applied research projects of the University.
This mind set must be internalized for all involved actors (e.g. researchers, students and teachers). 
The collaboration among the DSSCB, the Innovation Strategy of SDU and third-parties is crucial to maintain this innovative environment.




