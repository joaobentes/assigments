\section{Panel Discussions}

% First panel
Two main topics were debated in the first panel discussion: (i) communication in interdisciplinary research projects; and (ii) use of the term intervention. 
Regarding communication, the use of common set of terminologies tends to avoid misunderstandings among the actors in the project (e.g. physicians, patient, caregivers, IT professionals, nurses, psychologists etc). 
On the other hand, the complexity of adopting a homogeneous set of terminologies tends to proportionally increase due to the number specializations involved in the project. 
The term intervention is an interesting example of this issue. 
From the user side, intervention may bring the idea of forced or imposed behavior change, while from the psychologist point of view, it related to a technique to empower the user to change the behavior by himself/herself.
 
% Second panel
In the second panel, a broadly range of topics were discussed. 
The first was related to the importance of networking for researcher career, in particular for interdisciplinary projects like those related to health innovation. 
After that, the direction went towards the challenges of publishing papers in innovative environments.
One of the major concerns relied on the fact that some research outcomes have more impact into the society itself (e.g. new technique for bicycling, better ways of design running tracks etc) than into the scientific community.
Among all the suggestions proposed by the attendees, the most valuable was the fact that the feedback given by peer-reviewers can be used to improve research. 
This discussion also leaded to a definition of two types of research publication channels: (i) fast channel, i.e., social networks and newspapers; and (ii) slow channels, i.e., peer-review journals/conferences.


